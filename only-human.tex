\documentclass[11pt]{article}

\usepackage[a4paper, landscape, margin=0.5in]{geometry}
\usepackage{multicol}
\usepackage{fontspec}
\usepackage{parskip}
\usepackage{tabularx}
\usepackage{hyperref}
\usepackage{makecell}

\setmainfont{Economica}
\hypersetup{
    colorlinks=true,
    urlcolor=blue,
}

\renewcommand{\cellalign}{l}

\title{\vspace{-4ex}ONLY HUMAN 0.3.3\vspace{-6.5ex}}
\date{}

\begin{document}\begin{multicols}{3}
  \maketitle

  A simple tabletop RPG framework, for when you don't want superhuman
  characters. It requires a game master (GM) to run the world / non-player
  characters (NPCs), and at least $1$ player to run the player character(s)
  (PCs).

  To play, everyone should have d20s ($20$ sided dice), as well as paper,
  pencils, and erasers to record character details and notes. The GM should have
  a story, and details of relevant items, roles, and mechanics they want in the
  game (see the Still Human section for some examples).
  \section*{CORE RULES}

  \subsection*{Character Creation}

  Before the game, the players each create a character. To do so, they allocate
  $12$ points between four \textbf{stats}: Dexterity (DEX), Intelligence (INT),
  Strength (STR), and Will (WIL). Each \textbf{stat} may not be allocated more
  than $6$ points.

  Players also allocate $16$ points to \textbf{skills}. The GM may provide the
  option to pick a \textbf{role}, which confers an additional $4$ points that
  must be allocated within a particular set of \textbf{skills}. No skill can be
  allocated more than $4$ points.

  Each \textbf{skill} is associated with two \textbf{stats}. Below are some
  \textbf{skills}, but you can add other \textbf{skills} to the game, with the
  agreement of the GM and after associating the \textbf{skill} with a pair of
  \textbf{stats}.

  \begin{multicols}{2}
    DEX + DEX: ​\textit{Agility​},​ \textit{Sleight of hand}

    DEX + INT: \textit{​Ranged attacking​}

    DEX + STR: ​\textit{Speed}

    DEX + SPT: ​\textit{Stealth}

    INT + INT: ​\textit{Technology​},​ \textit{Medicine}

    INT + STR: ​\textit{Melee attacking}

    INT + SPT: \textit{Investigation}, \textit{Insight}

    STR + STR: \textit{​Athletics}

    STR + SPT: \textit{​Toughness}, \textit{Intimidation}

    SPT + SPT: ​\textit{Persuasion}
  \end{multicols}

  Keep in mind that couple of numbers should not totally describe your
  character. Think about their personality, about what they want, what they need
  (but might not know they need), and how they know the other characters in the
  group. Have secrets for your character, but want them to be discovered.

  \subsection*{Tests}

  When a character tries something with chance of both success and meaningful
  failure, the GM may have them take a \textbf{​test}​ in a \textbf{skill}: their
  controlling player (the GM for NPCs) rolls a d20, adds their \textbf{modifier}
  in the \textbf{skill}, and the GM compares that to a \textbf{difficulty} they
  have set. If the \textbf{test} result is at least the \textbf{difficulty} then
  the \textbf{test} is a success, otherwise it is a failure.

  To find the character's \textbf{skill} \textbf{modifier} add together their
  points in the associated \textbf{stats} and divide this by $2$ rounding down
  (this is just the \textbf{stat's} points for \textbf{skills} associated with
  the same \textbf{stat} twice) then add their points in the \textbf{skill}.
  References to a \textbf{skill} in a numerical context refer to this
  \textbf{modifier}.

  \subsection*{Contests}

  \textbf{Contests} are special kinds of \textbf{tests} for when two or more
  characters are simultaneously attempting tasks that come into conflict. Each
  character takes a \textbf{test} in the \textbf{skill} relevant to the task
  they are attempting. The character with the highest result succeeds, and the
  other character(s) fail (perhaps to varying degrees). The GM adjudicates the
  outcome of any ties.

  \subsection*{Turns}

  When things get tense characters start taking turns. The turn order, and
  whether or not they overlap, is up to the GM. Everyone taking a turn
  represents about 10 seconds of time. During a turn, characters may move, and
  once per turn (possibly during that movement) they may take an 
  \textbf{action​}. Characters may also talk at any point during turns, but
  players shouldn't communicate in-character more than is reasonable for the
  elapsed time.
  
  \textbf{Attacking}, \textbf{running}, \textbf{readying}, some object
  interactions, or trying to \textbf{stabalise} a character are all actions. As
  is anything else the GM deems to be.
  

  \subsection*{Moving}

  When characters move in their turn, they can move in a path of up to
  $10 + (2 \times Speed)$ metres. If a character is climbing, swimming,
  sneaking, carrying something bulky, moving over difficult ground, or otherwise
  slowed, then they are \textbf{restricted}. The distance a character moves
  whilst \textbf{restricted} counts twice against their movement.

  \subsection*{Actions}

  To \textbf{attack}, a character must know where they are attacking (if there
  is no target there the attack fails), be within their weapon's range of their
  target, and have nothing fully blocking their \textbf{attack}. The attacker
  then makes the relevant of either a \textit{​Ranged attacking​} or a
  \textit{​Melee attacking​​}  \textbf{test​}, adding their weapon's modifier. The
  \textbf{difficulty} for this ​\textbf{test​} is $10 + target\:DEX$. Increase the
  \textbf{difficulty} by $2$ for ranged attacks on targets within $2$ metres,
  and by $4$ if the attacker can see less than half of the target.

  If the attacker fails, nothing happens. Otherwise, the attack has an
  \textbf{impact} of the amount the \textbf{test} \textbf{difficulty} was
  exceeded by, minus the absorption of any armour the target is wearing. The
  target then takes a \textbf{difficulty} $15 + impact$ \textit{Toughness}
  \textbf{test}. If they fail they get \textbf{injured} and become
  \textbf{unstable}. If they fail by $10$ to $14$ they get \textbf{injured}
  twice, if they fail by $15$ or more they get \textbf{injured} three times.

  If a character \textbf{runs} they double the distance they are permitted to
  move in the current turn. If a character \textbf{readies} their player must
  specify a condition and an \textbf{action}. If the condition occurs before
  their next turn, their character immediately takes the action.

  \subsection*{Injury}

  Characters have a \textbf{health state} and are either \textbf{stable} or
  \textbf{unstable}. If a character gets \textbf{injured} they drop to the next
  \textbf{health state} down the table. If they are \textbf{unstable} in the
  same \textbf{health state} for more than $30$ minutes, their player rolls a
  d20. On an $11$ or higher they become \textbf{stable}, on a $10$ or lower they
  get \textbf{injured}. A successful \textbf{difficulty} $11$ \textit{Medicine}
  \textbf{test} will \textbf{stabilise} a character, but true recovery can
  require days or specialised equipment.

  \begin{tabularx}{\linewidth}{lX}
    Health State & Effect on Character \\
    \hline
    Normal & They're fine. Characters start in this state, \textbf{stable}. \\
    Wounded & They're hurt, they suffer a $-2$ penalty to all \textbf{tests}. \\
    Seriously Wounded & They're \textit{really} hurt, they suffer a $-4$ penalty
      to all \textbf{tests} and are \textbf{restricted}. \\
    Incapacitated & They're (effectively) unconscious. They cannot voluntarily
      do anything. You do not add their DEX to the \textbf{difficulty} of
      \textbf{tests} to attack them. \\
    Dead & They're dead, Jim.
  \end{tabularx}

  \section*{STILL HUMAN}

  Still Human is a sci-fi implementation of the ONLY HUMAN framework, that is,
  it provides \textbf{roles}, statistics for items, and a hint of a setting.
  Other implementations should be possible, but they are up to you to create.

  Still Human is designed to run games for teams of characters that are sent on,
  or somehow end up entangled in, various space-related missions in reasonably
  hard sci-fi near-ish future settings.

  \subsection*{Equipment}

  \begin{tabularx}{\linewidth}{lX}
    Armour (Absorption) & Description / Effects \\
    \hline
    Normal Clothes (0) & Stylish \textit{and} practical. \\
    Space Suit (2) & A surprisingly compact vacuum rated environmental suit. Two
      hot-swappable tanks (that passively re-fill in atmosphere) provide one
      hour of oxygen each. If the suit absorbs any damage it is torn, and must
      be repaired before it will be vacuum worthy. A character wearing a space
      suit is \textbf{restricted}. \\
    Reinforced Space Suit (3) & A military grade space suit. The same as a space
      suit, except it is only torn if its wearer is hit with an attack that
      results in an \textbf{impact} greater than $0$. \\
    Combat Armour (3) & Tactical body armour. \\
    Combat Carapace (4) & Heavy duty body armour. A character wearing combat 
      carapace is \textbf{restricted}.
  \end{tabularx}

  \begin{tabularx}{\linewidth}{lXX}
    Weapon (Type) & Range & Modifier \\
    \hline
    Combat Knife (Melee) & $2m$ & $+1$ \\
    Improvised Weapon (Melee) & $2m$ & $-1$ \\
    Shock Rod (Melee) & $2m$ & $+2$ \\
    Assault Rifle (Ranged) & $400m$ & $+3$ \\
    Auto Pistol (Ranged) & $50m$ & $+1$ \\
    Combat Knife (Ranged) & $10m$ & $+0$ \\
    Improvised Weapon (Ranged) & $10m$ & $-4$
  \end{tabularx}

  \begin{tabularx}{\linewidth}{lX}
    Other Items & Description / Effects \\
    \hline
    \makecell[t]{Short Range\\Communicator} & A small radio device. Provides
      encrypted communication with selected short range communicators configured
      to receive it within $20km$. \\
    \makecell[t]{Long Range\\Communicator} & A backpack sized radio device that
      provides encrypted communication to selected equipment configured to
      receive it within $40,000km$. \\
    Pad & A smartphone sized, portable, touch screen computer. \\
    Navigator & A small attachment to a pad that uses known maps, available 
      satellite data, and computer vision to attempt to inform the user of their
      position at all times. \\
    Scanner & A small attachment to a pad that allows various kinds of scanning.
      Where relevant, a scanner gives a $+2$ modifier to \textit{Investigation}
      \textbf{tests}. \\
    Medical Kit & A brief-case sized box containing various medical supplies.
      Where relevant, a medical kit gives a $+2$ modifier to \textit{Medicine}
      \textbf{tests}. Additionally, the medical kit contains $5$ stim-shots,
      that can be used to rouse an unconscious character to the seriously
      wounded \textbf{health state}, and a defibrillator that can be used to
      revive a dead character to the seriously wounded \textbf{health state} if
      they died in the last minute from causes that would be resolved by
      restarting their heart. The defibrillator has $5$ uses per charge. Use of
      both the stim-shots and the defibrillator require a successful
      \textbf{stabilisation} \textbf{action}, if the \textbf{test} is failed
      they are still used, but have no effect. \\
    Engineering Kit & A vacuum rated toolbox full of various equipment for
      dealing with the tech a space-faring engineer might expect to encounter.
      Where relevant, an engineering kit gives a $+2$ modifier to
      \textit{Technology} \textbf{tests}. Additionally, the engineering kit
      contains $5$ patches that can, as an \textbf{action}, be used to repair
      tears in space suits.
  \end{tabularx}

  \vfill\null
  \columnbreak

  \subsection*{Roles}

  In addition to the benefits of \textbf{roles} outlined in the character
  creation section of the basic rules, with the agreement of the GM a character
  that selects one of the following roles may also take the corresponding
  piece(s) of equipment, and all characters may take Normal Clothes, a Space
  Suit, a Short Range Communicator, a Pad, and their choice of either a combat
  knife or an Auto Pistol.

  Still Human also adds the \textit{Piloting} \textbf{skill} to a game. It is a
  DEX + INT \textbf{skill}, and is used in one of the following roles.

  \begin{tabularx}{\linewidth}{lXX}
    Role & Proficiencies & Equipment \\
    \hline
    Pilot & \textit{Piloting} and \textit{Investigation} & Navigator \\
    Communications
      & \textit{Technology}, \textit{Insight} and \textit{Persuasion}
      & Long Range Communicator \\
    Security
      & \textit{Ranged attacking}, \textit{Melee attacking}, and
        \textit{Toughness}
      & Reinforced Space Suit, Assault Rifle, and Shock Rod \\
    Engineering
      & \textit{Slight of hand}, \textit{Piloting} and \textit{Technology}
      & Engineering Kit \\
    Medical
      & \textit{Slight of hand} and \textit{Medicine}
      & Medical Kit \\
    Science
      & \textit{Technology} and \textit{Investigation}
      & Scanner
  \end{tabularx}

  \subsection*{Requirements and Requisitioning}

  An alternative to roles is to come up with a list of required proficiencies
  for the group, e.g. "We need people experienced in piloting and investigation,
  and someone trained in both technology and slight of hand", and let the group
  work out how they will fill those niches together. Similarly, a pool of
  equipment can be provided, for the group to share out. This could lead to
  relations forming between the characters whilst they are being created.

  \section*{GM TIPS}

  ONLY HUMAN is rather rules light. This is quite intentional, and is based on
  the idea that the fundamental core of an RPG is incredibly simple, and can
  benefit from being more exposed for GMs to tinker with. However, this power
  does also come with responsibility. Hence, this sections exists to provide
  various tips to try to help GMs out.

  \subsection*{Setting DCs}

  When setting DCs, generally keep in mind these difficulty levels:

  \begin{center}
    \begin{tabular}{ lr }
      Difficulty & DC \\
      \hline
      Very Easy & $5$ \\
      Easy & $10$ \\
      Moderately Complicated & $15$ \\
      Difficult & $20$ \\
      Very Difficult & $25$ \\
      Practically Impossible & $30$
    \end{tabular}
  \end{center}

  Unless you are intentionally trying to hide the difficulty of a task, if a
  character attempts something that is impossible to fail, impossible to
  successfully complete, or for which there is no meaningful cost to failure
  (the character can and would try again until they succeed), then there is no
  need for a \textbf{test}, simply narrate the outcome to the players.

  \subsection*{Turns}

  ONLY HUMAN does not prescribe a method for determining the order of turns,
  whether there even is an order to turns, or whether or not turns can overlap.
  This should generally be decided on a case by cases basis, as there is no one
  system that is good for every situation. The order could be determined by
  \textit{Speed} \textbf{contests}, it could just evolve from the scene,
  characters could be divided into teams that alternate, or perhaps everyone
  could go at once. The characters can be sequenced through any method you like.

  \subsection*{Adding to the Game}

  You should feel totally free to add items or rules to the game, it is
  intentionally simple for that reason. Keep in mind that it is expected for
  specific rules to override general rules, so don't be afraid of that. Perhaps
  you want a gun that fires so fast it's wielder effectively shoots twice, or an
  enemy that can take a lot more hits than a usual character. Don't worry if
  these things go against the general stuff in the core rules, GMing would be
  much less fun if the players knew how everything worked. This sections details
  some example additions.

  Adding equipment and roles (to effectively create your own implementation of
  ONLY HUMAN) is another great way to extend the game. Consider adding items or
  features that vary in more ways than the basic rules allow, such as a shotgun
  that can hit everything in a cone up to $10$ metres with a $+4$ or one target
  with a $+1$ out to $30$ metres.

  \subsection*{Antagonists and Enemies}

  When creating antagonists or enemies do not feel you need to stick to the
  options available to the player. You may want grunt alien enemies that drop
  dead on a single \textit{Toughness} \textbf{test} failure, or a villain
  antagonist so drugged up on stimulants that they are much harder to take down.
  You may want characters that have better or worse stats than the standard
  player character creation allows. Go for it.

  \subsection*{Character Equipment}

  Whilst there is no strict limit on how much equipment a character can carry,
  it is best to be realistic. If you think a PC is carrying too much, perhaps
  get the player to label where everything is on their character's body. You
  could also apply strict carry weight limits, such as $40 + 10 \times
  Athletics$ kilograms of equipment. If you wanted to be even more granular you
  could specify that a character carrying more than $40 + 5 \times Athletics$
  is \textbf{restricted}.

  \subsection*{Adjudicating Complex Tasks}

  ONLY HUMAN does not provide specific rules for adjudicating everything a
  player may wish to try. So, when dealing with non-standard tasks, attempt to
  find a way to distil them down into one or more core difficulties (parts of
  the task with a risk of meaningful failure and a chance of success) that can
  be related to \textbf{skills} (whether on the given \textbf{skills} table or
  not).

  Once you know what \textbf{skills} are involved, try to gauge the difficulty
  of these components of the task, work out if they happen in an order or at the
  same time, and work out what success or failure for individual parts or
  combinations of individual parts looks like. Finally, with these ideas in
  mind, have the character take \textbf{tests} based on the established skills
  and difficulties, in the established order, and provide the established
  outcomes.

  In practice, this does not need to be dealt with remotely as formally every
  time, but that structure forms guidelines for one way to react to players
  attempting complicated tasks.

  \subsection*{Contests}

  Be creative with your use of \textbf{contests}, they are not just for arm
  wrestles. Perhaps one character searching for another could be represented by
  a \textit{Investigation}/\textit{Stealth} \textbf{contest}, or one character
  trying to stop another from twisting out of their grip could be an
  \textit{Athletics}/\textit{Agility} \textbf{contest}.

  \subsection*{Progression}

  It is possible to implement a rudimentary progression system in ONLY HUMAN.
  The simplest way would be to give players more points to allocate to
  \textbf{stats} and \textbf{skills}, and perhaps raising the cap on the maximum
  allocatable to each. If you plan on doing this, consider starting the 
characters initially with fewer points than the character creation section
suggests, as increasing the available points too far beyond this point could
leave min-maxed characters with very large \textbf{modifiers}.

  Another simple progression mechanism, that is perhaps more inline with the
  ethos of ONLY HUMAN, would be to provide players with better equipment, and
  better connections in the setting of the game.

\end{multicols}\end{document}
